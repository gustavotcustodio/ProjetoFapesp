%%%%%%%%%%%%%%%%%%%%%%%%%%%%%%%%%%%%%%%%%%%%%%%%%%%%%%%%%%%%%%%%%%%%%%
\documentclass[12pt]{report}
\usepackage[a4paper]{geometry}
\usepackage[utf8]{inputenc}
\usepackage[english, portuguese]{babel}
\usepackage[myheadings]{fullpage}
\usepackage[T1]{fontenc}
\usepackage{fancyhdr}
\usepackage{graphicx, setspace}
\usepackage{sectsty}
\usepackage{url}
\usepackage{listings}

%%------
%% Comandos gerais
%% Observação: o arquivo "comandos.tex" tem que estar presente.
%%------
\input{comandos}
%
%%-----
%% Página de título
%% Observação: As definições que aparecem a seguir comporão a
%%             página de título e a folha de rosto.
%%-----
%% Define o nome da universidade onde o projeto foi desenvolvido.
\universidade{Universidade Federal do ABC}
%
%% Define o nome da faculdade onde o projeto foi desenvolvido.
\faculdade{Centro de Matemática, Computação e Cognição}
%
%% Define o título do projeto.
\titulo{Definir título do projeto @@@ }
%
%% Define a agencia de Fomento e a abreviatura. O primeiro argumento é o 
%% nome por extenso e o segundo a abreviatura.
%% Ambos os argumentos são obrigatórios
\agFomento{Fundação de Amparo à Pesquisa do Estado de São Paulo}{FAPESP}
%
%% Define o tipo de relatório. Pode ser Anual ou Final.
%% Não é obrigatório definir o tipo de relatório.
\tipoRelatorio{Final}
%
%% Define a modalidade de Projeto. Pode ser temático, regular, etc.
\modalidadeProjeto{Auxílio à Pesquisa Regular}
%
%% Define o autor do relatório.
\autor{Gustavo Torres Custodio}
%
%% Define a equipe do projeto (incluindo o pesquisador responsável no comando \membroA{}
\membroA{Gustavo Torres Custodio}
%% Inclua os demais membros do grupo (máximo +5)
\membroB{Debora Maria Rossi de Medeiros, Dra.}
%\membroC{Francisco}
%\membroD{Joao}
%\membroE{Antonio}
%\membroF{José}
%
%% Define o período da vigência do Projeto.
\periodoVigencia{01/junho/2015 a 30/maio/2017}
%
%% Define o período coberto pelo relatório.
\periodoRelatorio{01/junho/2015 a 30/maio/2017}
%
%% Define a cidade onde o projeto foi desenvolvido.
\cidade{Santo André}

%%-----
%% Página de título
%% Observação: Os comandos a seguir não devem ser mudados, 
%%             exceto caso necessário.
%%-----
\begin{document}
%
%% Define a numeração em romanos.
\pagenumbering{roman}
%
%% Gera a folha de título.
\geraTitulo
%
%% Gera a folha de rosto.
\folhaDeRosto
%
%% Escreva aqui o resumo em português.
\Resumo{
  Isso é um teste babalsbak das bsak blkab lsab dbaslkb saldbd lsab lasbdlsba ldbsa lbds lab ldkasbkldbs alk bdaslkb ldsakb lsdablk bald blkb dlskab lasdbl sablkds ablkb lkasbd lsdbaklb d 
  }
%
%% Escreva aqui o resumo em inglês.
\Abstract{
teste in english
}
%
%% Adicionará o sumário.
%% Mantenha o \thispagestyle{empty} e \clearpage
\tableofcontents
\thispagestyle{empty}
\clearpage
%
%% Define a numeração em arábicos.
\pagenumbering{arabic}

%%-----
%% Formatação do título da seção
%%-----
\sectionfont{\scshape}

%%-----
%% Corpo do texto
%%-----
\chapter{Resumo do projeto proposto}\label{chp:resumo_proj} 

Aqui começa o primeiro capítulo com o resumo do projeto proposto.

\chapter{Apresentação do Problema}\label{chp:apresentacao}

Aqui começa o terceiro capítulo com a descrição do apoio institucional recebido.

\chapter{Revisão Bibliográfica Sistemática}\label{chp:revisao}


% Uma revisão sistemática, assim como outros tipos de estudo de revisão, é uma forma de pesquisa que utiliza como fonte de dados a literatura sobre determinado tema. Esse tipo de investigação disponibiliza um resumo das evidências relacionadas a uma estratégia de intervenção específica, mediante a aplicação de métodos explícitos e sistematizados de busca, apreciação crítica e síntese da informação selecionada. As revisões sistemáticas são particularmente úteis para integrar as informações de um conjunto de estudos realizados separadamente sobre determinada terapêutica/intervenção, que podem apresentar resultados conflitantes e/ou coincidentes, bem como identificar temas que necessitam de evidência, auxiliando na orientação para investigações futuras5.

Uma revisão bibliográfica sistemática realiza um estudo na literatura científica sobre um determinado assunto aplicando sistematicamente métodos claramente definidos de busca, seleção e síntese de artigos \cite{sampaio2007}.

Um processo revisão sistemática pode ser sintetizada em 5 passos principais:

\begin{enumerate}
    \item Definir uma ou mais perguntas de pesquisa.
    \item Busca em bases de dados eletrônicas de periódicos científicos usando um conjunto de palavras chave.
    \item Inclusão e exclusão de artigos de acordo com critérios de inclusão e exclusão pré-definidos.
    \item Avaliação da qualidade metodológica dos artigos e crítica de resultados, descartando artigos que estejam abaixo de um nível de qualidade definido.
    \item Apresentação dos resultados, sintetizando os pontos principais dos artigos não descartados em nenhuma das etapas.
\end{enumerate}

No processo de revisão bibliográfica deste trabalho, primeiramente foi realizada uma busca em três diferentes repositórios de periódicos científicos utilizando uma \textit{query} de pesquisa. Alguns dos artigos encontrados foram descartados de acordo com o conteúdo de seus \textit{abstracts}, seguindo um conjunto de critérios de exclusão predefinidos. Os últimos restantes receberam uma pontuação relacionada à qualidade do artigo em relação às perguntas de pesquisa.

\section{Busca em Repositórios} \label{sec:busca_em_repositorios}

\textit{Science Direct}\footnote{Disponível em: \url{https://www.sciencedirect.com/}}, ACM\footnote{Disponível em: \url{https://www.acm.org/}} e IEEExplore\footnote{Disponível em: \url{https://ieeexplore.ieee.org/Xplore/home.jsp}} foram os repositórios escolhidos para a realização da busca por artigos. O \textit{Science Direct} foi escolhido pela sua diversidade de periódicos disponíveis, enquanto o IEEExplore e a ACM foram escolhidos pelo seu foco em computação e engenharia.

A busca nesses \textit{sites} foi feita utilizando a \textit{query}:

\begin{lstlisting}[language=Python]

(("evolutionary algorithms" OR "clustering algorithms")
 AND "classifier ensemble") OR ("evolutionary algorithms"
 AND ("weak learner" OR "weak classifier"))

\end{lstlisting}

\bigbreak

Na \textit{query} de busca busca-se encontrar a relação de algoritmos evolutivos ou algoritmos de agrupamento com \textit{ensembles} de classificadores. Procura-se também encontrar a relação entre classificadores fracos e algoritmos evolutivos, dado que essa é uma das perguntas de pesquisa escolhidas (Seção \ref{sec:perguntas_de_pesquisa}).

A busca retornou um total de 490 resultados. Desses, um total de 478 \textit{papers} restaram após a remoção de artigos repetidos e capítulos de livros encontrados.  

\section{Análise de Abstracts} \label{sec:analise_de_abstracts}

Uma avaliação prévia foi realizada nos 478 artigos restantes após a remoção de artigos duplicados e capítulos de livros. Nessa avaliação, os \textit{abstracts} dos artigos restantes são analisados e, se aqueles que apresentam algum dos critérios de exclusão definidos são descartados. Os critérios de exclusão definidos nessa revisão excluem:

\begin{itemize}
    \item documentos repetidos;
    \item documentos que não estão em sua versão final;
    \item artigos que não usam \textit{ensembles} para solução de problemas ou não propõem nenhum novo modelo de \textit{ensemble} de classificadores;
    \item documentos cujo idioma de escrita não é inglês nem português;
    \item artigos que abordam os conceitos de forma isolada, em nenhum momento combinando \textit{ensembles} com algoritmos evolutivos ou com algoritmos de agrupamento.
\end{itemize}

Um total de 119 documentos restaram após a eliminação de artigos que apresentavam um ou mais dos critérios citados. Após essa etapa, foi feita uma leitura completa nos artigos buscando responder um conjunto de perguntas de pesquisa.

\section{Perguntas de Pesquisa} \label{sec:perguntas_de_pesquisa}

As perguntas de pesquisa foram definidas buscando entender como otimização por algoritmos evolutivos e agrupamento de dados pode auxiliar na construção de \textit{ensembles} de classificadores.

O processo de revisão bibliográfica trabalhou com as seguintes perguntas:

\begin{enumerate}
    \item Quais algoritmos evolutivos são combinados com \textit{ensembles}?
    \item Quais benefícios de combinar esses algoritmos evolutivos com \textit{ensembles}?
    \item Quais técnicas de agrupamento de dados são combinadas com \textit{ensembles}?
    \item Quais benefícios de combinar essas técnicas de agrupamento com \textit{ensembles}?
    \item Quais critérios são utilizadas para combinar classificadores fracos?
\end{enumerate}

Um critério de pontuação foi definido de acordo com o número das perguntas acima que o trabalho responde, juntamente com os detalhes fornecidos em cada resposta.

% @@@ Menos de duas perguntas descarta. Se 3 perguntas não incluírem nem a 2 nem a 4 descarta também.
O critério de pontuação estabele um valor máximo de 8 pontos para cada artigo. Aqueles que obteram 5 ou mais pontos foram selecionados para a etapa de sintetização dos resultados, sendo que um artigo pode obter no máximo 8 pontos.

% @@@ Mais detalhes de como a pontuação é contabilizada incluindo threshold.
As perguntas 1 e 3 contabilizam 1 ponto cada quando suas respostas estão contidas no trabalho analisado, enquanto as perguntas 2, 4, e 5 fornecem 2 pontos no máximo cada. Nas perguntas 1 e 3 só é possível receber 0 ou 1 ponto, o que depende dos algoritmos evolutivos ou técnicas de agrupamento serem nomeados ou não. As demais perguntas podem receber valores intermediários entre 0 e 2 e são pontuadas de acordo com os seguintes critérios:

\begin{enumerate}
  \item[2] As vantagems de utilizar o algotimo evolutivo no trabalho é apresentada (1 ponto). A função \textit{fitness} do algoritmo evolutivo é descrita (0,5 ponto). O indivíduo/cromossomo que forma uma solução candidata no algoritmo evolutivo é descrito (0,5 ponto).
  \item[4] As vantages de utilizar a técnica de agrupamento no trabalho são apresentadas (1 ponto). É descrito como a técnica ajuda a construir o \textit{ensemble de classificadores}.
  \item[5] Os classificadores fracos que formam o \textit{ensemble} são listados (1 ponto). É descrito como os reultados dos classificadores fracos são combinados (1 ponto).
\end{enumerate}

% @@@ Quantos artigos sobraram e ligação com a próxima seção.
No total @@@ trabalhos conseguiram 5 pontos ou mais e compuseram a síntese de resultados.

\section{Síntese de Resultados} \label{sec:sintese_de_resultados}

A análise completa realizada nos @@@ trabalhos restantes encontrou algoritmos frequentemente utilizados para tarefas específicas na construção de \textit{ensembles}.

% @@@ Quantos trabalhos usaram algoritmos genéticos, fitness
A maioria dos algoritmos evolutivos utilizados para otimização de \textit{ensembles} foram Algoritmos Genéticos (AGs) comuns, ou então uma variação deles. Esses trabalhos buscaram utilizar AGs para otimizar tanto a acurácia quanto a diversidade dos \textit{ensembles} de classificadores. Na maioria desses trabalhos, os AGs têm a função de seleção de \textit{features}, isto é, selecionar quais atributos da base de dados serão utilizados para treinamento e teste dos classificadores.

% @@@ Seleção de features
No contexto de seleção de \textit{features}, os cromossomos dos AGs possuem comprimento igual ao número de atributos das instâncias, sendo que cada valor 0 ou 1 do cromossomo indica se o \textit{feature} correspondente foi selecionado para o processo de treinamento e teste dos classificadores base do \textit{ensemble}. A combinação de \textit{features} que resulta no melhor valor de \textit{fitness} é normalmente a selecionada para treinar todos os classificadores, no entanto alguns trabalhos utilizam as $n$ melhores combinações encontradas pelo AG, sendo que em cada um dos $n$ classificadores diferentes é utiizada uma das combinações de \textit{features} encontrada.

% @@@ Mais sobre outras tasks

% @@@ Algoritmos genéticos multi-objetivo acurácia vs diversidade
Muitos trabalhos utilizam como \textit{fitness} a acurácia da etapa de teste dos classificadores fracos, ou então uma métrica de diversidade entre os classificadores do \textit{ensemble}. Trabalhos que utilizam as duas métricas como \textit{fitness} @@@

% @@@ Clustering usado para separar itens em diferentes grupos e criar classificadores especializados.

\chapter{Proposta}

% @@@ Análise e fraquezas.

% @@@ Proposta focada em melhorar as falhas dos trabalhos revisados.


\chapter{Auxílios Anteriores}

O pesquisador responsável recebeu anteriormente uma bolsa TT-4 da Fapesp para o desenvolvimento de um trabalho junto à Agência de Inovação da UFABC @@@ 


%%-----
%% Referências bibliográficas
%%-----
\addcontentsline{toc}{chapter}{\bibname}
\bibliographystyle{abntex2-num}
\bibliography{bibliografia}

%%-----
%% Fim do documento
%%-----
\end{document}
